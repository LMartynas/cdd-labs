

\documentclass{beamer}
 
\usepackage[utf8]{inputenc}
 \usetheme{Madrid}
 \usecolortheme{beaver}
 \usefonttheme{structuresmallcapsserif}
 \usepackage{listings}
%Information to be included in the title page:


\title[The Map Pattern] %optional
{The Map Pattern}

\subtitle{A Short Introduction}

\author[Dr. Joseph Kehoe] % (optional, for multiple authors)
{Joseph Kehoe\inst{1}}

\institute[IT Carlow] % (optional)
{
	\inst{1}%
	Department of Computing and Networking\\
	Institute of Technology Carlow
}

\date[ITC 2017] % (optional)
{CDD101, 2017}

\logo{\includegraphics[height=1.5cm]{itcarlowlogo.png}}




 
 \AtBeginSection[]
 {
 	\begin{frame}
 		\frametitle{Table of Contents}
 		\tableofcontents[currentsection]
 	\end{frame}
 }
 
 
 
\begin{document}
 
\frame{\titlepage}
 
 
 
 \begin{frame}
 	\frametitle{Table of Contents}
 	\tableofcontents
 \end{frame}
 
 
 \section{Definition}
\begin{frame}
\frametitle{Definition}

\begin{itemize}
	\item The Map pattern replicates a function over every element of an index set.
	\item The function is applied to every element in the set concurrently.
	\item The index set may be abstract or associated with the elements of a collection.
	\item The function being replicated is called  an \emph{elemental function}.
\end{itemize}
\end{frame}

\begin{frame}
	\frametitle{Related Patterns}
Map is used for problems that are \emph{Embarrassingly Parallel}.

Often Combined with other patterns

\begin{itemize}
	\item Collectives often combined with map
	\begin{itemize}
		\item Gather
		\item Reduction
		\item Scan
	\end{itemize}
	\item Generalisations of Map:
	\begin{itemize}
		\item Stencil
		\item Convolution
		\item Recurrance
		\item Workpile
	\end{itemize}
\end{itemize}
\end{frame}

\begin{frame}
	\frametitle{Type of Concurrency}
\begin{itemize}
	\item If the elemental function contains no control flow then it is \emph{SIMD}
	\item If there is control flow it is  \emph{SPMD}
	\item Can also be \emph{SIMT} (\emph{SPMD} on tiled \emph{SIMD} hardware)
	\item If there are no side effects as a result of the elemental function then it is deterministic
	\item (This is good because?)
\end{itemize}
\end{frame}
\section{Examples}
\begin{frame}
	\frametitle{Example SAXPY}
SAXPY - Scaled Vector Addition

For each element i in a vector Y:
$Y[i] := A\times X[i]+Y[i]$
where $Y$ and $X$ are Vectors and $A$ is a constant

\begin{itemize}
	\item Depending on type of vector
	\item float (single precision) \textbf{SAXPY}
	\item Double \textbf{DAXPY}
	\item Complex float \textbf{CAXPY}
	\item Complex Double \textbf{ZAXPY}
\end{itemize}

The operation has a low arithmetic intensity (measure it!)

This implies it does not scale well (why?)
\end{frame}

\begin{frame}[fragile=singleslide]
	\frametitle{Serial Implementation}
\textbf{Basic Code}


\begin{lstlisting}
void saxpy(int n, float a,float y[], float x[])
{
#pragma omp parallel for
	for (int i=0; i < n; ++i)
	{
		y[i]=a * x[i] + y[i];
	}
}
\end{lstlisting}



\begin{itemize}
	\item Tiling will improve scalability
\end{itemize}
\end{frame}

\begin{frame}
	\frametitle{Example Mandelbrot Set}

\begin{itemize}
	\item Set of all points on plane c that do not go to infinity when $z=z^{2}+c$ is iterated (for ever)
	\item $z$ starts out at 0
	\item It has been shown that if the length of $z$ is greater than 2 then it is guaranteed to diverge
\end{itemize}
\end{frame}



\begin{frame}[fragile=singleslide]
	\frametitle{Mandelbrot Implementation}
	
	\textbf{Elemental function}
	\begin{lstlisting}
	int calc(Complex c, int depth)
	{
		int count=0;
		Complex z=0;
		for(int i=0;i<depth;++i)
		{
			if (abs(z)>2.0)
			{
				break;
			}
			z=z*z+c;
			count++;
		}
		return count;
	}
	\end{lstlisting}
\end{frame}

\begin{frame}[fragile=singleslide]
	\frametitle{Mandelbrot Implementation}
	
	\textbf{Main Loop}
	\begin{lstlisting}
	mandel( int p[][], int row, int col, 
	        int depth){
#pragma omp parallel for collapse(2)	  
	  for(int i=0;i<row,++i)
	  {
        for(int k=0;k<col;++k)
		{
	       p[i][k]=calc(Complex(i,k),depth);
		}
	  }
	}
	\end{lstlisting}
\end{frame}

\section{Optimisation}
\begin{frame}
	\frametitle{Sequence of Maps versus Map of Sequences}
\begin{itemize}
	\item A sequence of maps does not scale well (why?)
	\item To increase arithmetic complexity we must do more work between memory reads
so change into a  map of sequence (code fusion)
	\item We load all data at the start of the map
	\item Keep intermediate results in registers
	\item Write out final result at end
\end{itemize}
\end{frame}
\section{Related Patterns}
\begin{frame}
	\frametitle{Related Patterns}
\begin{description}
	\item[Stencil] each element reads in from surrounding elements but does not write to them
	\item[Convolution]	Stencil with weigths added to neighbouring elements
	\item[Workpile]	Work can grow dynamically as we are doing map
	\item[Divide and Conquor] Divide problem into smaller problems until base case can be solved serially
\end{description}
\end{frame}
 
 
 
 
\end{document}

